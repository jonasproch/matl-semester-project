\documentclass[12pt]{article}
\usepackage[utf8]{inputenc}
\usepackage{enumitem}
\usepackage{graphicx}

\title{Star Wars - The fat man strike's back}
\author{Jonáš Procházka}
\date{}
\renewcommand{\figurename}{Obrázek}

\begin{document}
\maketitle
\vfill{}
\noindent
Západočeská Univerzita V Plzni \hfill Jonáš Procházka\\
KKY/MATL \hfill 17.4.2024

\thispagestyle{empty}
\newpage
\setcounter{page}{1}

\section{Řešení}

\subsection{Grafické vyobrazení galaxie}
\begin{enumerate}[label=(\alph*)]
  \item \textbf{Zobrazení obrázku galaxie} \\
        Pro zobrazení obrázku galaxie jsem využil funkce \verb|imread|. Následně jsem použil funkci \verb|im2double| a \verb|rescale| pro převod na doubly a normalizaci obrázku. Nakonec již vytvořím figuru a zobrazím obrázek pomocí \verb|imshow|. \\
        \begin{figure}[h]
          \centering
          \includegraphics[scale=0.2]{resources/show_galaxy.png}
          \caption{Vyobrazený obrázek galaxie}
        \end{figure}
  \item \textbf{Rozšíření galaxie} \\
        Stejně jako u předchozího bodu jsem nejdříve obrázek načetl, převedl na double a přeškáloval. Poté jsem obrázek otočil o 90 stupňů pomocí funkce \verb|imrotate|. Konečně jsem si vytvořil kopii obrázku přetočenou z leva do prava pomocí \verb|fliplr|, obrázky spojil do jedné matice \\ \verb|[img_rotated img_flipped]| a zobrazil. \\
        \begin{figure}[h]
          \centering
          \includegraphics[scale=0.2]{resources/expanded_galaxy.png}
          \caption{Rozšířená galaxie}
        \end{figure}
\end{enumerate}

\end{document}
